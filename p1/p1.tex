\documentclass{article}

\usepackage[margin=1in]{geometry} 
\usepackage{amsmath,amsthm,amssymb}

\newcommand{\R}{\mathbb{R}}
\newcommand{\Z}{\mathbb{Z}}
\newcommand{\N}{\mathbb{N}}
\newcommand{\Q}{\mathbb{Q}}
\newcommand{\C}{\mathbb{C}}

\newenvironment{theorem}[2][Ejercicio]{\begin{trivlist}
\item[\hskip\labelsep{\bfseries #1}\hskip\labelsep{\bfseries #2.}]}{\end{trivlist}}
\newenvironment{lemma}[2][Lemma]{\begin{trivlist}
\item[\hskip\labelsep {\bfseries #1}\hskip\labelsep {\bfseries #2.}]}{\end{trivlist}}
\newenvironment{exercise}[2][Exercise]{\begin{trivlist}
\item[\hskip \labelsep {\bfseries #1}\hskip \labelsep {\bfseries #2.}]}{\end{trivlist}}
\newenvironment{problem}[2][Problem]{\begin{trivlist}
\item[\hskip \labelsep {\bfseries #1}\hskip \labelsep {\bfseries #2.}]}{\end{trivlist}}
\newenvironment{question}[2][Question]{\begin{trivlist}
\item[\hskip \labelsep {\bfseries #1}\hskip \labelsep {\bfseries #2.}]}{\end{trivlist}}
\newenvironment{corollary}[2][Corollary]{\begin{trivlist}
\item[\hskip \labelsep {\bfseries #1}\hskip \labelsep {\bfseries #2.}]}{\end{trivlist}}

\newenvironment{solution}{\begin{proof}[Solution]}{\end{proof}}

\begin{document}

\title{Métodos Numéricos de Optimización con restricciones.}
\author{Bustos Jordi\\Práctica I}

\maketitle

\begin{theorem}{1}
  Dada la función
  \[
    f(x,y) = (x - y^2)\left(x - \tfrac{1}{2}y^2\right).
  \]
  Mostrar que \( (0,0) \) es el único punto estacionario que NO es minimizador, sin embargo
  \[
    \forall d \in \mathbb{R}^n: \quad f(x^*) \leq f(x^* + td) \quad \text{para todo \( t \) pequeño}.
  \]
\end{theorem}

\begin{proof}
  Es claro que \( f \) es una función polinómica y por lo tanto es infinitamente diferenciable en todo \( \R^2 \) con: \begin{align*}
    \nabla f(x,y) & = \left( \frac{\partial f}{\partial x}, \frac{\partial f}{\partial y} \right) \\
                  & = \left( 2x - \tfrac{3}{2}y^2, -3xy + 2y^3 \right).
  \end{align*}
  Luego, \begin{align*}
    \nabla f(x,y) = 0 & \Leftrightarrow \begin{cases}
                                          2x - \tfrac{3}{2}y^2 = 0 \\
                                          -3xy + 2y^3 = 0
                                        \end{cases}
  \end{align*}
  Resolviendo el sistema se obtiene que el único punto estacionario es  \( (0, 0) \). Además, \begin{align*}
    H & = \frac{\partial^2 f(0, 0)}{\partial x^2} \cdot \frac{\partial^2 f(0, 0)}{\partial y^2} - {\left( \frac{\partial^2 f(0, 0)}{\partial (x, y)} \right)}^2 \\
      & 2 \cdot 0 - 0 = 0
  \end{align*}
  Por lo tanto, el criterio de la matriz hessiana no nos permite concluir si \( (0, 0) \) es un mínimo local, máximo local o punto silla. Si analizamos la función
  por el camino \( x = y^2 \) se obtiene que \( f(y^2, y) = \frac{1}{2} \cdot (y^4 - 3y^3) \) que tiene un máximo local en \( y = 0 \). Por otro lado, el camino
  \( y = 0 \) da \( f(x, 0) = x^2 \) que tiene un mínimo local en \( x = 0 \). Por lo tanto, \( (0, 0) \) no puede ser minimizador local. \\
  Sea \( d = (d_1, d_2) \in \R^2 \) y \( t > 0 \) suficientemente chico. Entonces \begin{align*}
    f(0, 0)                     & = 0                                                             \\
    f(t \cdot d_1, t \cdot d_2) & = t^2 (d_1^2 - \frac{3}{2} d_1 d_2^2 t + \frac{1}{2} t^2 d_2^4)
  \end{align*}
  Como los términos entre paréntesis son positivos para \( t \) suficientemente chico, se concluye que \( f(0, 0) \leq f(t \cdot d_1, t \cdot d_2) \).
\end{proof}

\vspace{0.25in}

\begin{theorem}{2}
  Sea
  \[
    f(x, y) = {(y - x^2)}^2 + x^5.
  \]
  Mostrar que \( f \) tiene un único punto estacionario que no es maximizador ni minimizador de \( f \).
\end{theorem}

\begin{proof}
  Análogamente al punto anterior calculemos el gradiente de \( f \): \begin{align*}
    \nabla f(x, y) & = \left( \frac{\partial f}{\partial x}, \frac{\partial f}{\partial y} \right) \\
                   & = \left( -4x(y - x^2) + 5x^4, 2(y - x^2) \right).
  \end{align*}
  Luego, \begin{align*}
    \nabla f(x, y) = 0 & \Leftrightarrow \begin{cases}
                                           -4x(y - x^2) + 5x^4 = 0 \\
                                           2(y - x^2) = 0
                                         \end{cases}
  \end{align*}
  Resolviendo el sistema se obtiene que el único punto estacionario es \( (0, 0) \). Además, \( H = 0 \) y por lo tanto el criterio de la matriz hessiana no nos permite concluir si \( (0, 0) \) es un mínimo local, máximo local o punto silla.
  Si analizamos la función por el camino \( y = x^2 \) se obtiene que \( f(x, x^2) = x^5 \) que toma valores tanto positivos como negativos en cualquier vecindad de \( x = 0 \). Por lo tanto \( (0, 0) \) no puede ser ni minimizador ni maximizador local.
\end{proof}

\vspace{0.25in}

\begin{theorem}{3}
  Si \(x^*\) es un minimizador local de \((PC)\) entonces \(x^*\) es un minimizador global de \(f\).
\end{theorem}

\begin{proof}
  Supongamos que \( x^* \) es minimizador local, pero no global.
  Entonces podemos encontrar un punto \( z \in \R^n \) con \( f(z) < f(x^*) \).
  Consideremos el segmento de línea que une \( x^* \) a \( z \), es decir,
  \begin{equation*}
    x = \lambda z + (1 - \lambda)x^*, \quad \text{para algún } \lambda \in (0,1].
  \end{equation*}

  Por la propiedad de convexidad de \( f \), tenemos
  \begin{equation*}
    f(x) \leq \lambda f(z) + (1 - \lambda) f(x^*) < f(x^*).
  \end{equation*}

  Cualquier vecindad \( \mathcal{N} \) de \( x^* \) contiene un trozo del segmento de línea,
  por lo que siempre habrá puntos \( x \in \mathcal{N} \) en los que se satisfaga la segunda ecuación.
  Por lo tanto, \( x^* \) no es un minimizador local.
\end{proof}


\vspace{0.25in}

\begin{theorem}{4}

  Sea \( f : \mathbb{R}^n \to \mathbb{R} \) diferenciable en \( \Omega \) convexo.
  Probar que \( f \) es convexa si y sólo si
  \begin{align*}
    \text{(a)} \quad & f(y) \geq f(x) + {(y-x)}^T \nabla f(x)
                     &                                                  & \forall x, y \in \Omega. \\
    \text{(b)} \quad & \nabla {f(x)}^T (y-x) \leq \nabla {f(y)}^T (y-x)
                     &                                                  & \forall x, y \in \Omega.
  \end{align*}
\end{theorem}

\begin{proof}
  Veamos (a) primero. Sea \( f \) convexa, para \(x, y \in \Omega \) y \( t \in (0, 1) \) fijo, definido \( d = y - x \) tenemos \( x + td \in \Omega \) y por la convexidad de \( f \) \begin{align*}
    f(x + td) = f\left((1-t)x + ty \right) \leq (1-t) f(x) + t f(y)
  \end{align*}
  Por lo tanto, \begin{align*}
    f(y) - f(x) & \geq \lim_{t \to 0^+} \dfrac{f(x+td) - f(x)}{ t} \\
                & = \nabla {f(x)}^T d                              \\
                & = \nabla {f(x)}^T (y - x)
  \end{align*}
  Para la vuelta, consideremos \( z = (1-t)x + ty \) y notemos que \begin{align*}
    f(x) \geq f(z) + \nabla {f(z)}^T (x-z) \text{ y } f(y) \geq f(z) + \nabla {f(z)}^T (y-z)
  \end{align*}
  Multiplicando la primera por \( (1-t) \) y la segunda por \( t \) obtenemos \begin{align*}
    (1-t) f(x) + t f(y) \geq f\left( (1-t)x + ty \right)
  \end{align*}
  Completando la demostración de (a). Para (b), consideremos primero que \( f \) es convexa y utilizando (a) obtenemos \begin{align*}
    f(y) & \geq f(x) + {(y-x)}^T \nabla f(x) \\
    f(x) & \geq f(y) + {(x-y)}^T \nabla f(y)
  \end{align*}
  Por lo tanto, \begin{align*}
    f(y) + f(x)           & \geq f(x) + f(y) + {(y-x)}^T \nabla f(x) + {(x-y)}^T \nabla f(y) \\
    0                     & \geq {(y-x)}^T \nabla f(x) + {(x-y)}^T \nabla f(y)               \\
    {(y-x)}^T \nabla f(y) & \geq {(y-x)}^T \nabla f(x)
  \end{align*}
  Recíprocamente, si \( f \) cumple (b), dados \( x, y \in \Omega \) consideremos \( \xi = (1-t)x + ty \) con \( t \in (0, 1) \). Entonces, \begin{align*}
    f(y) - f(x) = \nabla {f(\xi)}^T (y-x) \quad \text{ por el teorema del valor medio.}
  \end{align*}
  Si aplicamos la hipótesis a \( x, \xi \) se tiene que: \begin{align*}
    \nabla {f(x)}^T (\xi - x) & \leq \nabla {f(\xi)}^T (\xi - x)                                    \\
    \nabla {f(x)}^T (y - x)   & \leq \nabla {f(\xi)}^T (y-x) \quad \text{ ya que } \xi - x = t(y-x)
  \end{align*}
  Combinando con la igualdad del teorema del valor medio obtenemos \begin{align*}
    f(y) - f(x) & = \nabla {f(\xi)}^T (y-x) \geq \nabla {f(x)}^T (y-x) \\
    f(y)        & \geq f(x) + \nabla {f(x)}^T (y-x)
  \end{align*}
  Luego, vale (a) y por lo tanto \( f \) es convexa.
\end{proof}


\vspace{0.25in}

\begin{theorem}{5}
  Sea \( D \neq \emptyset \) un conjunto cerrado y convexo. Sea \( x \in \mathbb{R}^n \). Entonces:
  \begin{enumerate}
    \item Para todo \(y \in D\) se tiene que
          \[
            {(x - P_D(x))}^T (y - P_D(x)) \leq 0.
          \]

    \item Si \(\bar{z} \in D\) cumple que
          \[
            {(x - \bar{z})}^T (y - \bar{z}) \leq 0, \quad \forall y \in D
          \]
          entonces \(\bar{z} = P_D(x)\).

    \item La función \(P : \mathbb{R}^n \to D\) definida como \(P(x) = P_D(x)\) cumple la siguiente propiedad:
          \[
            \| P(x) - P(y) \| \leq \|x-y\|, \quad \forall x,y \in \mathbb{R}^n
          \]
          lo que implica que la proyección \(P\) es una función continua.
  \end{enumerate}
\end{theorem}

\begin{proof}
  Consideremos un punto arbitrario \( y \in D \). Dado \( t \in (0, 1) \), por la convexidad de \( D \), tenemos que \( (1 - t)P_D(x) + ty \in D \). Entonces,\begin{align*}
    \| x - P_D(x) \|  \leq \| x - (1-t) P_D(x) - ty \| = \| x - P_D(x) + t (P_D(x) - y) \|
  \end{align*}
  Luego,\begin{align*}
    \| x - P_D(x) \|^2 & \leq \|x - P_D(x) + t (P_D(x) - y) \|^2                                          \\
                       & = \|x - P_D(x) \|^2 + 2t {(x - P_D(x))}^T (P_D(x) - y) + t^2 \| P_D(x) - y  \|^2
  \end{align*}
  Como \( t > 0 \), tenemos que \( 2 {(x - P_D(x))}^T (y - P_D(x)) \leq t \|P_D(x) - y \|^2 \). Haciendo \( t \to 0 \) obtenemos \begin{align*}
    {(x - P_D(x))}^T (y - P_D(x)) \leq 0
  \end{align*}
  que es lo que queríamos demostrar. \\
  Supongamos que existe \( \bar{z} \in D \) que cumple la desigualdad anterior. Entonces, dado \( y \in D \) \begin{align*}
    \| x - \bar{z} \|^2 - \| x - y \|^2 & = -2 x^T \bar{z} + \bar{z}^T \bar{z} + 2 x^T y - y^T y      \\
                                        & = {( y - \bar{z})}^T (2x - y - \bar{z})                     \\
                                        & = {(y - \bar{z})}^T (2(x - \bar{z}) - (y - \bar{z})) \leq 0
  \end{align*}
  usando la igualdad algebraica \( \| a \|^2 - \| b \|^2 = {(a-b)}^T (a+b) \). Esto prueba que \( \bar{z} = P_D(x) \). \\
  Finalmente, para probar la tercera parte, consideremos \( p := P(x) \) y \( q := P(y) \). Aplicando la parte (1) convenientemente obtenemos \begin{align*}
     & {(x - p)}^T (q - p) \leq 0                      \\
     & {(y - q)}^T (p - q) \leq 0                      \\
     & \implies {(x-p)}^T(q-p) + {(y-q)}^T(p-q) \leq 0 \\
  \end{align*}
  Además \( {(y-q)}^T(p-q) = -{(y-q)}^T(q-p) \). Reordenando los términos obtenemos \begin{align*}
    {(x-p)}^T(q-p) - {(y-q)}^T(q-p) & \leq 0             \\
    {(x-y-(p-q))}^T(q-p)            & \leq 0             \\
    {(x-y-(p-q))}^T (p-q)           & \geq 0             \\
    {(x-y)}^T (p-q)                 & \geq \| p - q \|^2
  \end{align*}
  Recordemos que \( {(p-q)}^T (p-q) = \| p - q \|^2 \). Además, por la desigualdad de Cauchy-Schwarz, \( {(x-y)}^T (p-q) \leq \| x - y \| \cdot \| p - q \| \).
  Por lo tanto, \begin{align*}
    \| p - q \|^2 \leq \| x - y \| \cdot \| p - q \| \implies \| p - q \| \leq \| x - y \|
  \end{align*}
  que es lo que queriamos probar.
\end{proof}

\end{document}