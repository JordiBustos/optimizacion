\documentclass{article}

\usepackage[margin=1in]{geometry} 
\usepackage{amsmath,amsthm,amssymb}
\usepackage{graphicx}
\usepackage{tikz}
\usepackage{pgfplots}
\pgfplotsset{compat=newest}
\usepackage[spanish]{babel}
\usepackage{multicol}

\newcommand{\R}{\mathbb{R}}
\newcommand{\Z}{\mathbb{Z}}
\newcommand{\N}{\mathbb{N}}
\newcommand{\Q}{\mathbb{Q}}
\newcommand{\C}{\mathbb{C}}

\newenvironment{theorem}[2][Ejercicio]{\begin{trivlist}
\item[\hskip \labelsep {\bfseries #1}\hskip \labelsep {\bfseries #2.}]}{\end{trivlist}}
\newenvironment{lemma}[2][Lemma]{\begin{trivlist}
\item[\hskip \labelsep {\bfseries #1}\hskip \labelsep {\bfseries #2.}]}{\end{trivlist}}
\newenvironment{exercise}[2][Exercise]{\begin{trivlist}
\item[\hskip \labelsep {\bfseries #1}\hskip \labelsep {\bfseries #2.}]}{\end{trivlist}}
\newenvironment{problem}[2][Problem]{\begin{trivlist}
\item[\hskip \labelsep {\bfseries #1}\hskip \labelsep {\bfseries #2.}]}{\end{trivlist}}
\newenvironment{question}[2][Question]{\begin{trivlist}
\item[\hskip \labelsep {\bfseries #1}\hskip \labelsep {\bfseries #2.}]}{\end{trivlist}}
\newenvironment{corollary}[2][Corollary]{\begin{trivlist}
\item[\hskip \labelsep {\bfseries #1}\hskip \labelsep {\bfseries #2.}]}{\end{trivlist}}

\newenvironment{solution}{\begin{proof}[Solution]}{\end{proof}}

\sloppy
\begin{document}

\title{Métodos Numéricos de Optimización con restricciones.}
\author{Bustos Jordi\\Práctica IV}

\maketitle

\section*{Capítulo V}
\begin{theorem}{1}
    Elegir uno de los siguientes problemas y resolverlo, a partir del punto inicial mencionado, aplicando el método del gradiente proyectado con búsqueda de Armijo (\(\sigma_1 = 1/2\)) presentado previamente.
    \begin{itemize}
        \item Min \(f(x,y) = x - y\) sujeto a \(1 \leq x \leq 3\) y \( 1 \leq y \leq 2\), empezando en \(x_0 = (3,1)\).
        \item Min \(f(x,y) = x^2 + y^2 \) sujeto a \( 0 \leq x \leq 4 \) y \( 1 \leq y \leq 3 \), empezando en \(x_0 = (4, 3)\).
    \end{itemize}
\end{theorem}

\begin{proof}

\end{proof}

\begin{theorem}{2}
    Demostrar el Teorema 5.3: Para \(A \in \mathbb{R}^{m \times n}\), \(m < n\), \(\text{rango}(A) = m\), \(Q \in \mathbb{R}^{n \times n}\) simétrica y \(c \in \mathbb{R}^n\), si \(x^T Q x > 0 \, \forall x \neq 0 : Ax = 0\) entonces la matriz
    \[
        \begin{pmatrix}
            Q & A^T \\
            A & 0
        \end{pmatrix}
    \]
    es no singular. (Pensar en el absurdo)
\end{theorem}

\begin{proof}

\end{proof}

\begin{theorem}{3}
    En los siguientes problemas cuadráticos estudiar si se verifican las hipótesis del Teorema 5.3 y analizar en qué caso se puede afirmar que el problema tiene solución única.
    \begin{itemize}
        \item[(a)] Min \( 5x^2 + 4y^2 + 3z^2 + 4xy + 2xz - y + 4z \) sujeto a \( x + 2z = 4, \, y + 3z = 2 \).
        \item[(b)] Min \( 3x^2 + 6y^2 + 3z^2 - 4xy + 8xz + 4yz + 2x + 3y + z \) sujeto a \( x + z = 5, \, 2y + z = 1 \).
        \item[(c)] Min \( 6x^2 + 6y^2 + 5z^2 - 4xy - 2xz - 2yz + 3x - 2y - z \) sujeto a \( x + y = 1 \).
    \end{itemize}
\end{theorem}

\begin{proof}

\end{proof}

\begin{theorem}{4}
    Elegir uno de los siguientes problemas y resolverlo utilizando el método de restricciones activas comenzando desde los puntos indicados. Analizar el proceso gráficamente cuando sea posible.
    \begin{multicols}{2}
        \noindent
        \begin{align*}
            \text{(a)} & \quad \text{Min } {(x - 1)}^2 + {(y - 2.5)}^2 \\
                       & \quad \text{s.a } -x + 2y \leq 2              \\
                       & \quad x + 2y \leq 6                           \\
                       & \quad x - 2y \leq 2                           \\
                       & \quad x \geq 0, \, y \geq 0                   \\
                       & \text{Empezando en } x_0 = (2, 0)
        \end{align*}
        \columnbreak
        \begin{align*}
            \text{(b)} & \quad \text{Min } x^2 - xy + y^2 - 3x \\
                       & \quad \text{s.a } x + y \leq 4        \\
                       & \quad x \geq 0, \, y \geq 0           \\
                       & \text{Empezando en } x_0 = (0, 0)
        \end{align*}
    \end{multicols}
\end{theorem}

\end{document}